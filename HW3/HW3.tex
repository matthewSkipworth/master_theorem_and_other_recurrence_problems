\documentclass[page=4,11pt]{article}
\usepackage{amsmath}
\begin{document}
\ \\
Matthew Skipworth\\
TCSS 343\\
Homework3\\ \\
\textbf{2.1}\\ \\
1.) $$T(n)= \Bigg\{^{c,\ if\ n<8}_{16T(\frac{n}{8})+nlogn,\ if\ n\ \geq 8}$$
From the recurrence we see that: $a=16, b=8, f(n)=nlogn$.
We then plug the values into the function $n^{log_{b}a}$ and check
it against $f(n)$ for equality.
$$nlogn\ ?\ n^{log_{8}16},\ nlogn<n^{\frac{3}{4}}$$
Based on these conditions we know that the recurrence fits case 1, 
where: $T(n)= \Theta(n^{log_{b}(a)-\epsilon})$ or $T(n)=  \Theta(n^{log_{\frac{4}{3}}})$.\\ \\
2.) $$T(n)= \Bigg\{^{c,\ if\ n<8}_{2T(\frac{n}{8})+\sqrt[3]{n},\ if\ n\ \geq 8}$$
From the recurrence we see that: $a=2,\ b=8,\ f(n)=n^{1/3}$.
We then plug the values into the function $n^{log_{b}a}$ and check
it against $f(n)$ for equality.
$$n^{ \frac{1}{3}}\ ?\ n^{log_{8}2},\ n^{ \frac{1}{3}}=n^{ \frac{1}{3}}$$
Based on these conditions we know that the the recurrence fits case 2, where: $T(n)= \Theta(n^{log_{b}a}logn)$ or $T(n)=\Theta(n^{\frac{1}{3}}logn)$\\
3.) $$T(n)= \Bigg\{^{c,\ if\ n<2}_{3T(\frac{n}{2})+9^{n},\ if\ n\ \geq 2}$$
From the recurrence we see that: $a=3, b=2, f(n)=9^n$.
We then plug the values into the function $n^{log_{b}a}$ and check
it against $f(n)$ for equality.
$$9^n\ ?\ n^{log_{2}3},\ 9^n>n^{1.58496 \dots}$$
Based on these conditions we know that the recurrence fits case 3, 
where: $T(n)= \Theta(f(n))$ or $T(n)=  \Theta(9^n)$.\\ \\
4.) $$T(n)= \Bigg\{^{c,\ if\ n\leq1}_{3T(\frac{3n}{5})+n^{2},\ if\ n > 1}$$
From the recurrence we see that: $a=3, b=\frac{5}{3}, f(n)=n^{2}$.
We then plug the values into the function $n^{log_{b}a}$ and check
it against $f(n)$ for equality.
$$n^{2}\ ?\ n^{log_{\frac{5}{3}}3},\ n^2<n^{2.15066 \dots}$$
Based on these conditions we know that the recurrence fits case 1, 
where: $T(n)= \Theta(n^{log_ba})$ or $T(n)=  \Theta(n^{2.15066 \dots})$.\\ \\
5.) $$T(n)= \Bigg\{^{c,\ if\ n\leq1}_{3T(\frac{3n}{5})+n^{2}\sqrt{n},\ if\ n > 1}$$
From the recurrence we see that: $a=3, b=\frac{5}{3}, f(n)=n^{2}\sqrt{n}$.
We then plug the values into the function $n^{log_{b}a}$ and check
it against $f(n)$ for equality.
$$n^{2.5}\ ?\ n^{log_{\frac{5}{3}}3},\ n^{2.5}>n^{2.15066 \dots}$$
Based on these conditions we know that the recurrence fits case 3, 
where: $T(n)= \Theta(f(n))$ or $T(n)=  \Theta(n^{2.5})$.\\ \\
\textbf{2.2}\\ \\
1.) $$T(n)= \Bigg\{_{3T(\frac{n}{2})+n^3\ for\ n>1}^{c\ for\ n \leq 1}$$
2.) From the recurrence we see that $a=3,b=2,f(n)=n^3$. We then plug the values into the function $n^{log_{b}a}$ and check it against f(n) for equality. 
$$n^3\ ?\ n^{log_{2}3},\ n^3>n^{1.58496 \dots}$$
Based on these conditions we know that the recurrence fits case 3, where: $T(n)= \Theta (f(n))= \Theta(n^3)$.\\ \\ \\
3.)$$T(n)= \Bigg\{_{T(\frac{n}{2})+nlog(logn)\ for\ n>1}^{c\ for\ n \leq 1}$$
4.) From the recurrence we see that $a=1,b=2,f(n)=nlog(logn)$. We then plug the values into the function $n^{log_{b}a}$ and check it against f(n) for equality. 
$$nlog(logn)\ ?\ n^{log_{2}1},\ nlog(logn)>n^0$$
Based on these conditions we know that the recurrence fits case 3, where: $T(n)= \Theta (f(n))= \Theta(nlog(logn))$.\\ \\
%\begin{align*}
%3+8&=7\\
%7&=7+0
%\end{align*}
\end{document}